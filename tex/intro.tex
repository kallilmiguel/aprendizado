\section{Introdução}
O termo \emph{Machine Learning} se refere à detecção, por parte do computador, em reconhecer padrões em uma base de dados.
Nas últimas décadas, tornou-se uma ferramenta fundamental para qualquer tarefa que requer a extração de informações de grandes conjuntos de dados. 

Estamos cercados por tecnologia baseada em \emph{Machine Learning}: Mecanismos de pesquisa aprendem como nos disponibilizar os melhores resultados, softwares anti-spam aprendem como filtrar nossos e-mails, assim como transações de cartões de créditos são mantidas em segurança por um software que aprende a detectar fraudes. \cite{shalev2014understanding}
Para tal, vários algoritmos conseguem fazer com que um computador possa automaticamente detectar os padrões de uma base de dados. 

Nesse trabalho será utilizada uma base de dados criada para identificar uma voz como masculina ou feminina baseada nas propriedades acústicas da voz e da fala. A base de dados consiste em 3 168 amostras de voz gravadas, coletadas de falantes masculinos e femininos, tendo 22 características, todas envolvendo a frequência sonora da fala, sendo algumas delas:

\begin{itemize}
	\item Frequência média (kHz)
	\item Desvio padrão da frequência
	\item Mediana da frequência (kHz)
	\item Centróide da frequência 
\end{itemize}

Para o reconhecimento dos padrões dessa base de dados serão utilizados duas técnicas: Regressão Linear e Máquinas de Vetores de Suporte.



